\section*{Resultados de aprendizaje}

\large\textbf{Conceptuales}\par

\begin{itemize}
	\item Explica los conceptos, razones, clasificación, el modelo de referencia OSI y el sistema de protocolos TCP/IP utilizadas en las redes de computadoras.
	\item Conoce las topologías de red, dispositivos de interconexión y estándares de red.
	\item Aprende y realiza pruebas de conexiones utilizando utilerías de conectividad.
	\item Conoce de las normas, características y componentes de un cableado estructurado, como infraestructura para organizar y administrar los cables de una red de área local.
\end{itemize}

\large\textbf{Procedimentales}\par

\begin{itemize}
	\item Identifica, enfrenta y resuelve problemas de cominicación entre ordenadores.
	\item Diseña una red de computadoras, proponiendo soluciones a través de los dispositivos de interconexión de redes, protocolos de comunicaciones y de control y asegurando la red a través de medidas de seguridad.
	\item Aplica normas, para la elaboración de un cableado estructurado, como infraestructura para organizar y administrar una red de área local.
\end{itemize}

\large\textbf{Actitudinales}\par

\begin{itemize}
	\item Respeta la diversidad cultural, su cosmovisión, a su cultura, a los derechos humanos individuales, colectivos y al proceso de autonomía de las Regiones Autónomas de la Costa Caribe de Nicaragua.
	\item Asume los ejes transversales de URACCAN\: interculturalidad, género, derechos humanos, emprendimiento y desarrollo empresarial en sus labores y prácticas cotidianas.
	\item Propicia y cultiva valores tales como: el compañerismo, la disciplina, la puntualidad, la honestidad y el respeto mutuo durante el proceso de enseñanza-aprendizaje.
\end{itemize}