\section*{Contenidos}

\large\textbf{Distribución de horas}
\begin{center}
	\resizebox{\textwidth}{!}{
		\begin{tabular}{|c|l|c|c|c|}
			\hline
			\textbf{No.}           & \multicolumn{1}{c|}{\textbf{Unidad}}    & \textbf{Horas teóricas} & \textbf{Horas prácticas} & \textbf{Horas de estudio independiente} \\ \hline
			I                      & Introducción a las redes y conectividad & 4                       & 4                        & 16                                      \\ \hline
			II                     & Transmisión de datos                    & 4                       & 4                        & 16                                      \\ \hline
			III                    & Tecnología de comunicación en red       & 4                       & 4                        & 16                                      \\ \hline
			IV                     & Cableado estructurado                   & 4                       & 8                        & 24                                      \\ \hline
			V                      & Seguridad en redes                      & 4                       & 4                        & 16                                      \\ \hline
			VI                     & Diseño e implementación de una red      & 4                       & 12                       & 32                                      \\ \hline
			\multicolumn{1}{|l|}{} & Evaluaciones                            & 0                       & 4                        & 8                                       \\ \hline
			\multicolumn{1}{|l|}{} & \textbf{TOTAL}                          & \textbf{24}             & \textbf{40}              & \textbf{128}                            \\ \hline
		\end{tabular}
	}
\end{center}
\vspace{0.5cm}

\large\textbf{Contenidos}\\\\
\textbf{Unidad I: Introducción a las redes y conectividad}
\begin{itemize}
	\item Introducción:
	      \begin{itemize}
		      \item ¿Qué es una red de ordenadores?
		      \item Uso y razones de las redes de computadoras.
	      \end{itemize}
	\item Elementos de una red.
	\item Tipos de redes:
	      \begin{itemize}
		      \item Local Area Network (LAN).
		      \item Personal Area Network (PAN).
		      \item Metropolitan Area Network (MAN).
		      \item Wide Area Network (WAN).
		      \item Redes troncales.
		      \item Redes inalámbricas.
		      \item Redes domésticas.
		      \item Interredes.
	      \end{itemize}
	\item Topología de redes:
	      \begin{itemize}
		      \item Tipos de topología.
		      \item Topología en estrella.
		      \item Topología en bus.
		      \item Topología en anillo.
		      \item Topología en malla.
		      \item Topologías híbridas.
	      \end{itemize}
	\item Dispositivos de transmisión e interconexión de redes.
	\item Modelos de referencia:
	      \begin{itemize}
		      \item Modelo de referencia OSI.
		      \item Capas del modelo OSI.
		      \item Modelo de referencia TCP/IP.
		      \item Capas del modelo TCP/IP.
		      \item Comparación entre los modelos de referencia OSI y TCP/IP.
	      \end{itemize}
	\item Estándares de red:
	      \begin{itemize}
		      \item Proyecto 802 Conexión.
		      \item 802.2 Conexión entre redes.
		      \item 802.2 Control de enlace lógico (LLC).
		      \item 802.3 Ethernet.
		      \item 802.4 Token Bus.
		      \item 802.5 Token Ring.
		      \item 802.6 FDDI.
		      \item 802.11 LAN Inalámbricas.
	      \end{itemize}
\end{itemize}\vspace{0.5cm}

\textbf{Unidad II: Transmisión de datos}
\begin{itemize}
	\item Conceptos y terminología.
	\item Transmisión de datos analógicos y digitales.
	\item Ancho de banda:
	      \begin{itemize}
		      \item Multiplexación.
		      \item Ensanchado.
	      \end{itemize}
	\item Medios de transmisión:
	      \begin{itemize}
		      \item Guiados.
		      \item No guiados.
		      \item Inalámbricos:
		            \begin{itemize}
			            \item El espectro electromagnético.
			            \item Radiotransmisión.
			            \item Microondas.
			            \item Infrarrojos.
			            \item Ondas de luz.
		            \end{itemize}
	      \end{itemize}
	\item Redes telefónicas y cableadas.
	\item Redes inalámbricas.
	\item Redes sobre VoIP.
	\item Virtualización.
\end{itemize}\vspace{0.5cm}

\textbf{Unidad III: Tecnología de comunicación en red}
\begin{itemize}
	\item Funciones de la capa de enlace.
	\item Elementos de la capa de enlace.
	\item Subcapas de enlace de datos:
	      \begin{itemize}
		      \item Subcapa de enlace lógico (LLC).
		      \item Subcapa de enlace físico (MAC).
	      \end{itemize}
	\item Tramas:
	      \begin{itemize}
		      \item Formato de la trama.
	      \end{itemize}
	\item Funciones de la subcapas de datos:
	      \begin{itemize}
		      \item Subcapa de enlace lógico (LLC).
		            \begin{itemize}
			            \item Detección y corrección de de errores.
			            \item Control de enlace de datos.
		            \end{itemize}
	      \end{itemize}
	\item LAN cableada: Ethernet.
	      \begin{itemize}
		      \item Tipos de redes cableadas.
	      \end{itemize}
	\item LAN inalámbrica.
	\item Conexión LAN, redes troncales y LAN virtuales.
	\item WAN inalámbricas: telefonía móvil y redes por satélite.
	\item SONET / SDH.
	\item Redes de circuito virtual: Frame Relay y ATM.
	\item Tecnología a nivel de red:
	      \begin{itemize}
		      \item Direccionamiento lógico.
		      \item Protocolo de Internet (IP).
		      \item Procesos básicos de la capa de red:
		            \begin{itemize}
			            \item Direccionamiento.
			            \item Encapsulación.
			            \item Enrutamiento.
			            \item Desencapsulación.
		            \end{itemize}
		      \item Funcionamiento interno de la capa de red:
		            \begin{itemize}
			            \item Datagramas.
			            \item Circuitos virtuales.
		            \end{itemize}
		      \item Servicios de la capa de red:
		            \begin{itemize}
			            \item Servicios orientados a la conexión.
			            \item Servicios no orientados a la conexión.
			            \item Configuración de red.
			            \item Parámetros de configuración.
			            \item Asignación de IP dinámica y estática.
			            \item Dirección IP.
			            \item Clases de direccionamiento IP.
			            \item Máscara de subred.
			            \item Puerta de enlace o Gateway.
			            \item Servidor DNS.
			            \item Protocolo IPv4 e IPv6.
			                  \begin{itemize}
				                  \item Protocolo IPv4.
				                  \item Protocolo IPv6.
			                  \end{itemize}
		            \end{itemize}
	      \end{itemize}
	\item Tecnología a nivel de transporte y aplicación:
	      \begin{itemize}
		      \item Nivel de transporte:
		            \begin{itemize}
			            \item Comunicación proceso a proceso:
			                  \begin{itemize}
				                  \item UDP.
				                  \item TCP.
				                  \item SCTP.
			                  \end{itemize}
			            \item Control de congestión y calidad del servicio.
		            \end{itemize}
		      \item Nivel de aplicación:
		            \begin{itemize}
			            \item Servicios WWW, HTTP.
			            \item Servicios DNS.
			            \item Servicios DHCP.
			            \item SNPM/POP3.
		            \end{itemize}
	      \end{itemize}
\end{itemize}\vspace{0.5cm}